\documentclass[12pt]{article}

\usepackage{sbc-template}
\usepackage{graphicx,url}
\usepackage[utf8]{inputenc}

\usepackage{verbatim}
\usepackage{xspace}
\newcommand{\FC}       {Freechains\xspace}
\newcommand{\reps}     {\emph{reps}\xspace}
\newcommand{\onerep}   {\emph{1~rep}\xspace}
\newcommand{\nreps}[1] {\emph{#1~reps\xspace}}
\newcommand{\code}[1]  {\texttt{\footnotesize{#1}}}
\newcommand{\Xon} {$1{\rightarrow}N$\xspace}
\newcommand{\Xno} {$1{\leftarrow}N$\xspace}
\newcommand{\Xnn} {$N{\leftrightarrow}N$\xspace}
\newcommand{\Xoo} {$1{\leftrightarrow}1$\xspace}
\newcommand{\Xo}  {$1{\hookleftarrow}$\xspace}

\renewcommand{\theenumi}{\alph{enumi}}

\hyphenation{off-line}

\sloppy

\title{
    \emph{AM-Tools} - Operating Permissionless P2P JSON Datasets
}

\author{Anonymous}
\address{Anonymous}

\begin{document}

\begin{comment}
- 8 paginas
- Descrição e motivação do problema resolvido pela ferramenta;
- Arquitetura da solução e descrição das principais funcionalidades;
- Descrição da demonstração planejada para o Salão de Ferramentas, informando equipamentos necessários para tal;
- URL onde a ferramenta está disponível (obrigatório na modalidade Código Aberto);
- URL dos manuais e documentação da ferramenta, incluindo informações e requisitos para instalação (obrigatório na modalidade Código Aberto);
- URL com um vídeo explicando a instalação e as funcionalidades da ferramenta (obrigatório na modalidade Código Fechado e opcional na modalidade Código Aberto).
\end{comment}

\maketitle

\begin{abstract}
Networked collaborative applications, such as Google Docs and Github, allow
users to work together concurrently.
These applications rely on distributed datasets that represent documents, code,
and other data, which users expect to see and share in a consistent way.
However, most practical collaborative systems rely on centralized authorities
to ensure data consistency and correctness.
%
In this work, we propose a set of tools to read and write JSON datasets in a
permissionless P2P environment, in which users participate in a consensus
mechanism to ensure consistency and correctness.
%
\emph{AM-Tools} reconcile Automerge, a JSON-based conflict-free replicated data
type, with Freechains, a permissionless blockchain with a reputation mechanism
to moderate content and prevent abuse.
%
\emph{AM-Tools} resembles a distributed version control system (DVCS), but with
a consensus mechanism to merge conflicting editions while preserving data
correctness.
\end{abstract}

\section{Introduction}
\label{sec.introduction}

Networked collaborative applications allow remote users to share projects while
working together concurrently \cite{wu2010partial}.
Examples of collaborative applications include Google
Docs~\cite{attebury2008google}, Github~\cite{preston2008github} and
Wikipedia~\cite{wikipedia2004wikipedia}.
%
These applications rely on distributed datasets that are read, written, stored,
and synchronized through the network.
By datasets, we mean any kind of data structures that can represent documents,
code, or social media feeds, as found in collaborative applications.
Users expect to see and share these datasets in a consistent way, even if their
machines synchronize sporadically and at different times.
%
In this work, we use generic JSON files to represent datasets, given that they
are human readable and widely adopted as an interchange format.

Currently, most practical collaborative systems rely on centralized authorities
to ensure data consistency and correctness.
However, centralized authorities concentrate too much power, since they control
data ownership and service availability~\cite{pincheira2022decentralized}.
As an emerging alternative, P2P systems~\cite{androutsellis2004survey} aim to
decentralize control, such that applications grow organically with new
users~\cite{rodrigues2010peer}, who contribute with storage, availability, and
also, in our context, with moderation of the datasets.

However, P2P systems strive to ensure data consistency and correctness,
particularly in the presence of malicious users or
Sybils~\cite{douceur2002sybil}.
For instance, malicious users may abuse the system by vandalizing datasets with
SPAM or hate speech.
By consistency, we mean that all replicas reach the same state, while by
correctness, we mean that such consistent state must also be immune to
vandalism (integrity) and preserve the users intent
(accuracy)~\cite{litt2022peritext}.
Integrity and accuracy inevitably depend on the subjective judgement of users,
which we refer as \emph{subjective correctness}.

As a solution to preserve consistency and accuracy is to rely on conflict-free
replicated data types (CRDTs)~\cite{shapiro2011conflict}.
As an example, Automerge~\cite{kleppmann2018automerge} provides JSON data
structures as CRDTs, such that they can be read and written concurrently
without conflicts.
Nevertheless, CRDTs are still subject to corner cases, such as simultaneous
editions of the very same part of documents.
In addition, CRDTs are not themselves immune to malicious users and Sybils,
which might vandalize datasets with no restrictions.
In such cases, a consensus mechanism might still be necessary to preserve data
integrity.

As a partial solution to preserve integrity, permissioned P2P systems opt for a
centralized authority to grant membership to new users.
Examples of permissioned P2P networks include Ripple~\cite{schwartz2014ripple}
and Hyperledger Fabric~\cite{androulaki2018hyperledger}.
%
Nevertheless, our goal is for a permissionless environment that requires no
centralized membership authority, preserving the organic growth of traditional
P2P systems.

In a permissionless context, a major challenge is to support concurrent data
edition and moderation, such that users share consistent and correct datasets
even in the presence of Sybils.
%
Bitcoin~\cite{nakamoto2008bitcoin} is a Sybil-resistant permissionless protocol
because it is expensive to write to its unique timeline (either via
proof-of-work or transaction fees).
However, Bitcoin and cryptocurrencies in general provide no subjective means to
moderate conflicts in collaborative datasets, but instead rely on tokens for
arbitrate conflicts.
%
As an alternative targeting collaborative applications,
Freechains~\cite{sant2020freechains} resists Sybils through a reputation
mechanism that moderates content through likes and dislikes and, at the same
time, delivers network consensus.
However, unlike CRDTs, Freechains does not support structured datasets with
attached semantics, since it represents content as a simple linked list of
binary blobs stored in individual ``chains'', which each application must
interpret appropriately.

In this work, we propose \emph{AM-Tools}, a set of tools that reconciles
Automerge and Freechains, extending JSON CRDTs with a consensus mechanism that
can handle consistency corner cases and also moderate malicious editions:
Automerge editions are ordered by the Freechains consensus mechanism providing
a priority to resolve conflicts, thus preserving accuracy.
In addition, further moderation can revert priorities or even remove abusive or
malicious editions, thus preserving integrity.
%
A dataset is structured as Automerge JSON file stored on Freechains.
Each file is a separate blockchain that stores a list of modifications to the
JSON.
The list is traversed from the beginning in consensus order to recreate the
file as a complete JSON.
%
In this sense \emph{AM-Tools} resembles a distributed version control system
(DVCS), but with a consensus mechanism to merge conflicting editions while
preserving data correctness.

\emph{AM-Tools} is composed of 4 utilities: \code{AM-rw}, \code{AM-diff},
\code{AM-patch}, and \code{AM-freechains}.
%
\code{AM-rw} allows to read and write portions of a JSON file.
\code{AM-diff} obtains the differences between two JSON files.
\code{AM-patch} applies differences to a JSON file.
\code{AM-freechains} commits and checkouts a JSON file to its corresponding
blockchain.

The rest of this paper is organized as follows.
Section~\ref{sec.both} describes the basic functionality of Automerge and
Freechains as separate tools.
Section~\ref{sec.amtools} describes \emph{AM-Tools} and how it combines
Automerge and Freechains.

\section{Automerge and Freechains}
\label{sec.both}

This section describes the basic functionality of Automerge and Freechains in
separate.

\subsection{Automerge}

Automerge~\cite{kleppmann2018automerge} is an open-source library that provides
CRDT-based JSON datasets to build collaborative applications.
%
Automerge is designed for \emph{local-first software}~\cite{TODO} which, unlike
cloud-based software, stores and operates the datasets locally, such that
applications can work normally while offline.
%
Therefore, users can make concurrent changes to a shared JSON without
centralized coordination:
When users synchronize, Automerge ensures that changes are merged automatically
in a principled manner.
%
More specifically, Automerge is an operation-based CRDT (CmRDT~\cite{TODO})
that models JSON datasets as append-only logs of concurrent commutative
modifications, such as the insertion or removal of list
elements~\cite{kleppmann2017conflict}.
%
When users synchronize, these logs are merged to reach the exactly same state
according to the eventual consistency model~\cite{TODO}.

%This is the main reason why concurrent changes are inevitable, and why CRDTs

The basic JavaScript API of Automerge is as follows:

\begin{itemize}
\item \code{obj = AM.init()}: \\
    Initializes a JSON as an empty object \code{obj=\{\}}.
\item \code{obj2 = AM.change(obj1, f)}: \\
    Modifies object \code{obj1} through function \code{f}, resulting in
    object \code{obj2}.
\item \code{chgs = AM.getChanges(obj1, obj2)}: \\
    Compares objects \code{obj1} and \code{obj2}, resulting in changes
    \code{chgs}.
\item \code{obj2 = AM.applyChanges(obj1, chgs)} \\
    Applies changes \code{chgs} to object \code{obj1}, resulting in object
    \code{obj2}.
\item \code{bin = AM.save(obj)} \\
    Converts object \code{obj} to a byte array \code{bin}.
\item \code{obj = AM.load(bin)} \\
    Converts a byte array \code{bin} to object \code{obj}.
\end{itemize}

Note that the basic API of Automerge is network agnostic and does not provide
a mechanism to synchronize objects, requiring extra libraries for this purpose.

The API usage is straightforward:
    initialize a dataset with \code{AM.init},
    make local changes with \code{AM.change} and \code{AM.getChanges},
    synchronize with other users (extra),
    apply remote changes with \code{AM.applyChanges},
    persist locally with \code{AM.save} and \code{AM.load}.
Except for function \code{AM.change}, the other functions are straightforward.
%
Because Automerge is a CmRDT and must keep track of all operations, JSON
objects cannot be manipulated directly in JavaScript.
Instead, \code{AM.change} receives a function argument to manipulate
the given object with tracking enabled.

The next example simulates two peers \code{A} and \code{B} manipulating a
JSON concurrently to reach the final state \code{\{ list: ['A','B']\}}:

\noindent
{\footnotesize
\begin{minipage}[t]{0.6\textwidth}
\begin{verbatim}
// PEER A
A1 = AM.init()
A2 = AM.change(A1, json => {
    json.list = []
})

X0 = AM.getChanges(A1,A2)   --sync-X0-->

A3 = AM.change(A2, json => {
    json.list.push('A')
})

XA = AM.getChanges(A2,A3)   <--XB--XA-->
A4 = AM.applyChanges(A3, XB)
\end{verbatim}
\end{minipage}
\begin{minipage}[t]{0.4\textwidth}
\begin{verbatim}
// PEER B
B1 = AM.init()




B2 = AM.applyChanges(B1, X0)

B3 = AM.change(B2, json => {
    json.list.push('B')
})

XB = AM.getChanges(A2,A3)
B4 = AM.applyChanges(B3, XA)
\end{verbatim}
\end{minipage}
}

Peers \code{A} and \code{B} initialize a new JSON independently (\code{A1} and
\code{B1}).
Then, peer \code{A} adds an empty list (\code{A2}), computes the difference
(\code{X0}), and synchronizes it with peer \code{B} (\code{B2}).
Then, both peers push values to the list concurrently (\code{A3} and
\code{B3}), and synchronizes the differences (\code{XA} and \code{XB}), which
are applied in the other sides (\code{A4} and \code{B4}).
Automerge ensures that the final states \code{A4} and \code{B4} are exactly the
same, with the list containing both values \code{'A'} and \code{'B'}.
However, since the editions are concurrent, the final list may become either
\code{['A','B']} or \code{['B','A']}, depending on arbitrary criteria such as
peer identity or timestamp.
Nevertheless, regardless of the order, the users intent to push two values into
the list is preserved.

Now, consider that instead of calls to \code{json.list.push}, the peers set the
lists directly as \code{json.list=['A']} and \code{json.list=['B']}.
In this case, the concurrent assignments are incompatible, and an
irreconcilable conflict occurs.
Nevertheless, Automerge arbitrarily chooses one of the assignments, but still
detects and indicates the conflict through the API \code{AM.getConflicts}.
Note that for such corner cases, Automerge does not preserve accuracy, since
the intent of one of the users is not satisfied.

In addition, since Automerge does not rule over peer synchronization, it cannot
enforce data integrity either.
For instance, a priori, nothing stops malicious users from vandalizing the
datasets with SPAM or erroneous information.
%
Therefore, even though Automerge provides a solid foundation for
eventually-consistency JSON datasets, it still cannot ensure integrity and
accuracy for all cases.

\subsection{Freechains}

%\bibliographystyle{abbrv}
%\bibliography{freechains}

\end{document}
